\documentclass{beamer}

\mode<presentation> {

%\usetheme{default}
%\usetheme{AnnArbor}
%\usetheme{Antibes}
%\usetheme{Bergen}
%\usetheme{Berkeley}
%\usetheme{Berlin}
%\usetheme{Boadilla}
\usetheme{CambridgeUS}
%\usetheme{Copenhagen}
%\usetheme{Darmstadt}
%\usetheme{Dresden}
%\usetheme{Frankfurt}
%\usetheme{Goettingen}
%\usetheme{Hannover}
%\usetheme{Ilmenau}
%\usetheme{JuanLesPins}
%\usetheme{Luebeck}
%\usetheme{Madrid}
%\usetheme{Malmoe}
%\usetheme{Marburg}
%\usetheme{Montpellier}
%\usetheme{PaloAlto}
%\usetheme{Pittsburgh}
%\usetheme{Rochester}
%\usetheme{Singapore}
%\usetheme{Szeged}
%\usetheme{Warsaw}

% As well as themes, the Beamer class has a number of color themes
% for any slide theme. Uncomment each of these in turn to see how it
% changes the colors of your current slide theme.

%\usecolortheme{albatross}
%\usecolortheme{beaver}
%\usecolortheme{beetle}
%\usecolortheme{crane}
%\usecolortheme{dolphin}
%\usecolortheme{dove}
%\usecolortheme{fly}
%\usecolortheme{lily}
%\usecolortheme{orchid}
%\usecolortheme{rose}
%\usecolortheme{seagull}
%\usecolortheme{seahorse}
%\usecolortheme{whale}
%\usecolortheme{wolverine}

%\setbeamertemplate{footline} % To remove the footer line in all slides uncomment this line
%\setbeamertemplate{footline}[page number] % To replace the footer line in all slides with a simple slide count uncomment this line

%\setbeamertemplate{navigation symbols}{} % To remove the navigation symbols from the bottom of all slides uncomment this line
}

\usepackage{graphicx} % Allows including images
\usepackage{booktabs} % Allows the use of \toprule, \midrule and \bottomrule in tables

%----------------------------------------------------------------------------------------
%	TITLE PAGE
%----------------------------------------------------------------------------------------

\title[Estimation of Nutritional Value of Food]{Estimation of Nutritional Value of Food in Real Time Using Deep Learning} % The short title appears at the bottom of every slide, the full title is only on the title page

\author [Team No. 17]  
{Akshay \textbf{(69)} Jithinraj \textbf{(30)}  Midhun \textbf{(37)}  Jishnu \textbf{(70)}} % Your name
\institute[Dept. of CSE, GCEK] % Your institution as it will appear on the bottom of every slide, may be shorthand to save space
{
Guided by: \textbf{Prof. Shine S}  
\bigskip
\\S7 CSE (2016 Batch)\\ % Your institution for the title page
}
\date{\today} % Date, can be changed to a custom date

\begin{document}

\begin{frame}
\titlepage % Print the title page as the first slide
\end{frame}

\begin{frame}
\frametitle{Outline} % Table of contents slide, comment this block out to remove it
\tableofcontents % Throughout your presentation, if you choose to use \section{} and \subsection{} commands, these will automatically be printed on this slide as an overview of your presentation
\end{frame}

%	PRESENTATION SLIDES
%------------------------------------------------
\section{Introduction} 

%\subsection{} % A subsection can be created just before a set of slides with a common theme to further break down your presentation into chunks
\begin{frame}
\frametitle{Introduction}
\begin{itemize}
    \item High Calorie food intake can be harmful and result in obesity,
which is a preventable medical condition that causes abnormal accumulation of fat in the body.
    \item  It can result in numerous diseases such as diabetes, cholesterol, heart attacks,blood pressure and other diet related diseases.
\item Method helps in determining the nutritional content of food automatically by making it feasible for a person to learn about what food might contain and how healthy it might be.
\item The inherent theme is to automatically detect food items from an image of a platter and then estimate the respective food attributes such as the percentage of calcium, iron etc.
\end{itemize}
\end{frame}

\section{Motivation}
\begin{frame}
\frametitle{Motivation}
\begin{itemize}
    \item Rapid increase in dietary diseases such as cholesterol, blood pressure, strokes etc during the last few decades caused by unhealthy food routine.
    \item If people become more aware about their food intake and its nutritional value, then the diseases mentioned above and allergies can be reduced.
\end{itemize}
\end{frame}

\section{Literature Review}
\begin{frame}
\frametitle{Literature Review}
\begin{itemize}
    \item Ege and Yanai proposed a system that directly estimate food calories from photos of food by simultaneously learning about food categories and ingredients.\cite{a2}
    \item Pouladzadeh proposed a system that involves capturing an image of the food and processing it through predefined steps, which follow a pipe line architecture.\cite{a3}
\end{itemize}
\end{frame}


\section{Objectives}
\begin{frame}
\frametitle{Objectives}
\begin{itemize}
    \item Use CNNs to recognize the food item in an image.
    \item Estimates food attributes using text retrieval from internet archives as well as scrapping of data from nutritional and recipe websites for ingredients and nutrient counts.
    \item Data is trained on a two layer neural network, from which we can compute probabilities of existing ingredients in a particular food item.
\end{itemize}
\end{frame}

\section{Relevance and Impact on Society}
\begin{frame}
\frametitle{Relevance and Impact on Society}
\begin{itemize}
    \item Diet management can made easy.
    \item Diseases such as high cholesterol, blood pressure, strokes etc which caused by unhealthy food routine can be reduced.
\end{itemize}
\end{frame}


\section{Conclusions}
\begin{frame}
\frametitle{Conclusions}
\begin{itemize}
    \item We proposed a system to provide health information about the food we eat.
    \item Estimate approximate ingredients and nutritional values in food.
    \item Can record real time images of food and analyze it for nutritional content, so that people can improve their dietary habits and lead a healthy life.
\end{itemize}
\end{frame}

%\begin{frame}
%\frametitle{References}
%\footnotesize{
%\begin{thebibliography}{99} % Beamer does not support %BibTeX so references must be inserted manually as below
%\bibitem[Smith, 2012]{p1} John Smith (2012)
%\newblock Title of the publication
%\newblock \emph{Journal Name} 12(3), 45 -- 678.
%\end{thebibliography}
%}
%\end{frame}

\begin{frame}[allowframebreaks] %allow to expand references to multiple frames (slides)

\frametitle{References}

\scriptsize{\bibliographystyle{plain}}

\bibliography{ref} %bibtex file name without .bib extension
\nocite{*}
\end{frame}

\begin{frame}
\Huge{\centerline{Thank You}}
\end{frame}

\end{document} 